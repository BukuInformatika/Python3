
\subsection{Teori}
\begin{enumerate}
\item Pada python variabel tidak perlu dideklarasikan, pendeklarasian terjadi secara otomatis pada saat memberikan suatu nilai atau data ke variabel. Terdapat beberapa jenis tipe data variabel pada python, diantaranya :
				\begin{itemize}
\item Python Numbers, dimana akan menyimpan data yang berupa angka. Penggunaan pada python sebagai berikut : 
					\begin{lstlisting}
					var1 = 5
					var2 = 48.9
					\end{lstlisting}
	
\item Python Text, dimana akan menyimpan data yang berupa teks ataupun karakter. Penggunaan pada python harus diapitkan oleh tanda petik ("..."), contohnya :
					\begin{lstlisting}
					nama = "Irvan"
					jnskelamin = "L"
					\end{lstlisting}
					
\item Python Boolean, dimana yang hanya memiliki 2 nilai yaitu True dan False saja. penggunaan pada python huruf pertama harus kapital, contohnya :
					\begin{lstlisting}
					var3 = True
					var4 = False
					\end{lstlisting}
				\end{itemize}

			\item \begin{itemize}
					\item Meminta input pada user
					nama = input("Masukkan Nama Anda : ")
					
					\item menampilkan output
					print "Hello Nama Saya Adalah",nama
				\end{itemize}

			\item \begin{itemize}
					\item Operator tambah
					\begin{lstlisting}
					a = b + c
					\end{lstlisting}

					\item Operator kurang
					\begin{lstlisting}
					a = b - c
					\end{lstlisting}

					\item Operator kali
					\begin{lstlisting}
					a = b * c
					\end{lstlisting}

					\item Operator bagi
					\begin{lstlisting}
					a = b / c
					\end{lstlisting}

					\item Konversi integer ke string
					\begin{lstlisting}
					konvVar = str(var1)
					\end{lstlisting}

					\item Konversi string ke integer
					\begin{lstlisting}
					konvVar = int(var2)
					\end{lstlisting}
				\end{itemize}

			\item \begin{itemize}
				\item Pengulangan for, kemampuan mengulang proses data menggunakan urutan apapun, seperti list.
				contoh penggunaan pada Python dan contoh kode adalah :

					\begin{lstlisting}
					for i in range(10):
						print(i)
					\end{lstlisting}
					
				\item Pengulangan while, kemampuan mengulang proses data yang akan terus berlanjut jika kondisinya True.
				contoh penggunaan pada Python dan contoh kode adalah :
					\begin{lstlisting}
					i= 0
					while i < 10 :
						i=i+1
						print ("loop ke =", i)
					\end{lstlisting}
				\end{itemize}
				
			\item Pengambilan keputusan berguna untuk menentukan tindakan apa yang akan diambil sesuai dengan kondisi yang ada. Contohnya :
				\begin{lstlisting}
				nilai = 9
				if(nilai > 7):
					print("Selamat Anda Lulus")
				else:
					print("Maaf Anda Tidak Lulus")
				\end{lstlisting}
				
				Dan untuk kondisi di dalam kondisi contohnya :
				
				\begin{lstlisting}
				gaji = 10000000
				berkeluarga = True
				if gaji > 3000000:
					print "Gaji sudah diatas UMR"
					if berkeluarga:
							print "Wajib ikutan asuransi dan menabung untuk pensiun"
						else:
							print "Tidak perlu ikutan asuransi"
				else:
					print "Gaji belum UMR"
				\end{lstlisting}

			\item \begin{itemize}
					\item Syntax Errors, Salahnya dalam penulisan sintaks.
					cara penanganannya adalah dengan menganalisa bagian kode yang error dan memperbaiki sintaks tersebut.
					
					\item Exceptions, error yang terjadi karena sintaks tidak dapat dieksekusi.
					cara penanganannya adalah dengan menganalisa bagian kode yang error dan memperbaiki sintaks tersebut.
				\end{itemize}
			
			\item Try Except adalah cara penanganan error pada Python.
			Contohnya : 
				\begin{lstlisting}
				x = 0
				try:
					x = 1 / 0
				except Exception, e:
					print e
				\end{lstlisting}

		\end{enumerate}
		
	\subsection{Keterampilan Pemrograman}
		\begin{enumerate}
			\item \lstinputlisting{src/chapter2/1174043_1.py}

			\item \lstinputlisting{src/chapter2/1174043_2.py}

			\item \lstinputlisting{src/chapter2/1174043_3.py}

			\item \lstinputlisting{src/chapter2/1174043_4.py}

			\item \lstinputlisting{src/chapter2/1174043_5.py}

			\item \lstinputlisting{src/chapter2/1174043_6.py}

			\item \lstinputlisting{src/chapter2/1174043_7.py}

			\item \lstinputlisting{src/chapter2/1174043_8.py}

			\item \lstinputlisting{src/chapter2/1174043_9.py}

			\item \lstinputlisting{src/chapter2/1174043_10.py}
			
			\item \lstinputlisting{src/chapter2/1174043_11.py}
		\end{enumerate}
		
	\subsection{Keterampilan Penanganan Error}
		\begin{enumerate}
			\item TypeError yaitu error di dalam tipe data disaat melakukan substring dan ingin memasukkannya ke dalam kondisi for 
			yang hanya menerima tipe int. jadi harus merubah tipe inputan yaitu string menjadi integer.

			\item \lstinputlisting{src/chapter2/1174043_2err.py}
		\end{enumerate}